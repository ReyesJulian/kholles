\documentclass{article}

\date{27 Janvier 2024}
\usepackage[nb-sem=16, auteurs={Hugo Vangilluwen}]{../kholles}

\begin{document}
	\maketitle
 
	\begin{question_kholle}
		{Soit $f$ définie sur un intervalle contenant $x_0$ \\
		$f$ est dérivable en $x_0$ $\iff$ $f$ admet un $DL_1(x_0)$\\
		De plus, dans ce cas, ce $DL_1(x_0)$ est... }
  
		$\Rightarrow$ Montrons que :$f(x)=f(x_0)+f'(x_0)(x-x_0)+o(x-x_0)$\\
		$f$ derivable sur un intervalle contenant $x_0$ donc $f'(x_0)$ existe.\\
  
		Calculon:\\
		$\lim\limits_{x \to x_0} \frac{f(x)-f(x_0)-f'(x_0)(x-x_0)}{x-x_0}$\\
		$\frac{f(x)-f(x_0)-f'(x_0)(x-x_0)}{x-x_0}=\frac{f(x)-f(x_0)}{x-x_0}-f'(x_0) \to f'(x_0)-f'(x_0)=0$\\
		Ainsi:$f(x)=f(x_0)+f'(x_0)(x-x_0)+o(x-x_0)$\\\\
		$\Leftarrow$ Soit $f$ une fonction admentant un$DL_1$ en $x_0$, Montrons qu'elle est derivable en $x_0$:\\
		$f$ admet un $DL_1$ en $x_0$ donc:\\
		$\exists (a,b)\in \mathbb{R}:f(x)=a+b(x-x_0)+o(x-x_0)$\\
		on a donc:$f(x)=a+o(1)\to a$\\
		Ainsi $f$ est continue en $x_0$, donc $\lim\limits_{x \to x_0}f(x)=f(x_0)$\\
		Par unicité de la limite: $a=f(x_0)$\\
		$\frac{f(x)-f(x_0)}{x-x_0}=\frac{f(x_0)+b(x-x_0)+o(x-x_0)-f(x_0)}{x-x_0}=b+\frac{o(x-x_0)}{x-					x_0}\underste{x\rightharrow x_0}{\rightharrow} b$\\
		Donc $f$ est derivable en $x_0$


	\end{question_kholle}

	
	\begin{question_kholle}
		{Deux fonctions équivalentes au voisinage de $a$ ont le même signe sur un voisinage de $a$}
		
		Soient $f : \mathcal{D} \rightarrow \R$ et $g : \mathcal{D} \rightarrow \R$ telles que $f(x) \underset{x \rightarrow a}{\sim} g(x)$ avec $a \in \mathcal{D}$. \\
		Appliquons la définition de l'équivalence pour $\epsilon \leftarrow \frac{1}{2}$, il existe un voisinage $V$ de $a$ tel que :
		\begin{equation*}
			\forall x \in V \cap \mathcal{D},
			| f(x) - g(x) | \leqslant \frac{1}{2} | g(x) |
		\end{equation*}
	
		Fixons un tel voisinage $V$.
		Nous obtenons :
		\begin{equation*}
			\forall x \in V \cap \mathcal{D},
			\underbrace{g(x) - \frac{1}{2} | g(x) |}_{\text{du signe de }g(x)}
			\leqslant f(x) \leqslant
			\underbrace{g(x) + \frac{1}{2} | g(x) |}_{\text{du signe de }g(x)}
		\end{equation*}
	
		Ainsi $f(x)$ et $g(x)$ ont le même signe sur $V \cap \mathcal{D}$.
	\end{question_kholle}

	\begin{question_kholle}
		[ Soient $f \in \Cont{\infty}{\mathcal{D}}{}$ et $a \in \overset{\circ}{\mathcal{D}}$. Supposons que $E_0 = \left\{ p \in \N^* \setminus \{1\} \;|\; f^{(p)}(a) \neq 0 \right\}$ est non vide. \\
		Posons $p_0 = \min E_0$. \\
		$f$ admet un extremum local en $a$ si et seulement si $f'(a) = 0$ et $p_0$ est pair. \\
		$f$ admet un point d'inflexion en $a$ si et seulement si $p_0$ est impair. ]
		{Condition nécessaire et suffisante pour qu'une fonction \Cont{\infty}{}{} admette un extremum local ou un point d'inflexion}
		
		Soient de tels objets. Traitons le cas de l'extremum local.
		
		\noindent $f \in \Cont{\infty}{}{}$ donc, la formule Taylor-Young donne un $DL_{p_0}(a)$ de $f$ :
		\begin{equation*}
			f(x) \underset{x \rightarrow a}{=}
			\sum_{k=0}^{p_0} \frac{f^{(k)}(a)}{k!} (x-a)^k + o \left( (x-a)^{p_0} \right)
		\end{equation*}
		
		En développant :
		\begin{equation*}
			f(x) \underset{x \rightarrow a}{=}
			f(a) + \underbrace{f'(a)}_{= 0} + \underbrace{\ldots + \frac{f^{(p_0-1)}(a)}{(p_0-1)!} (x-a)^{p_0-1}}_{= 0 \text{ par défintion de }p_0} + \frac{f^{(p_0)}(a)}{p_0!} (x-a)^{p_0} + o \left( (x-a)^{p_0} \right)
		\end{equation*}
		
		Ainsi (car $f^{(p_0)}(a) \neq 0$)
		\begin{equation}
			f(x) - f(a) \underset{x \rightarrow a}{\sim} \frac{f^{(p_0)}(a)}{p_0!} (x-a)^{p_0}
		\end{equation}
		Au voisinage de $a$, $f(x) - f(a)$ et $\frac{f^{(p_0)}(a)}{p_0!} (x-a)^{p_0}$ ont le même signe.
		\\
		
		Supposons que $f$ admette un extremum local en $a$.
		Or $a \in \overset{\circ}{\mathcal{D}}$ et $f$ est dérivable en 0, donc $f'(a) = 0$.
		Comme $f$ admette un extremum local en $a$, $f(x) - f(a)$ est de signe constant au voisinage de $a$.
		Donc $\frac{f^{(p_0)}(a)}{p_0!} (x-a)^{p_0}$ est de signe constant au voisinage de $a$.
		Par conséquent, $p_0$ est pair.
		\\
		
		Réciproquement, supposons que $f'(a) = 0$ et que $p_0$ est pair. $\frac{f^{(p_0)}(a)}{p_0!} (x-a)^{p_0}$ est de signe constant au voisinage de $a$. Donc $f(x) - f(a)$ est de signe constant au voisinage de $a$. Ainsi, $a$ est un extremum local de $f$.
		\\
		
		Traitons le cas du point d'inflexion. La formule de Taylor-Young donne :
		\begin{equation}
			f(x) - \underbrace{\left( f(a) + (x-a)f'(a) \right)}_{\text{tangente en } (a,f(a))}
			\underset{x \rightarrow a}{\sim} \frac{f^{(p_0)}(a)}{p_0!} (x-a)^{p_0}
		\end{equation}
		Le signe de l'écart courbe/tangente en $a$ est donc celui de $\frac{f^{(p_0)}(a)}{p_0!} (x-a)^{p_0}$. Ce qui conclut de la même manière que l'extremum local.
	\end{question_kholle}
	
\end{document}
